\documentclass[a4paper,beeees]{hwhProblem}
\usepackage[utf8]{inputenc}
\usepackage[T1]{fontenc}
\usepackage{amsmath}
\usepackage{amssymb}
\usepackage{hyperref}
\usepackage{listings}
\usepackage{tikz-cd}
\usepackage{booktabs}
\usepackage{pgfplots}
\usepackage{float}
% macros
\DeclareMathOperator{\hot}{Hot}
\newcommand{\rnaturals}{\left\{1,2,\dotsc,r\right\}}
\newcommand{\ProjK}[1]{\text{Proj}K[{#1}_0,\dotsc,{#1}_n]}
\newcommand{\funciso}[2][F]{#1 \simeq{} #1 \times{} e_{#2}}

\pgfplotsset{compat=1.16}

\DTMsavedate{publishdate}{2019-05-04}

\begin{document}
\title{Writing la text}
\author{zavzav\#1083}
\date{\DTMusedate{publishdate}}
\maketitle{EV-080308-CP-4}{IV}{\LaTeXe}
\section*{Introduction to LaTeX}
\LaTeX{} is a powerful typesetting language used to design documents. Unlike many other word processors (such as Microsoft Word or LibreOffice Writer), \LaTeX{} is written in a plain text format which is then compiled to other formats (usually PDF). Its advantages include higher programmability, more professional output, easier embeddings into other template styles and easy extensibility with the use of user-created libraries. Its implementations are available on most operating systems, usually free and open source.

To install \LaTeX{} on Windows systems, you should install MiKTeX or TeXLive (make sure to choose the complete installation with all the packages). A nice editor for \LaTeX{} on Windows is TeXworks and TeXstudio but any text editor works and is often the better choice once you customise it and get used to it (on Windows, I recommend Visual Studio Code). However it is easier to start with if you use a dedicated \LaTeX{} editor such as the two mentioned before.

To install \LaTeX{} on Unix-like systems, use the built-in package manager. Usually the packages are named pdflatex and texlive. Create your \LaTeX{} file using any text editor and compile with \texttt{pdflatex yourfile.tex} which will result in a file \texttt{yourfile.pdf}. More advanced users can also try XeTeX which features more options.
\subsection*{Resources}
The challenge is meant for people with any knowledge of \LaTeX{} (or lack of thereof). The use of search engines is allowed and encouraged, sites you should look out for the most are \href{https://tex.stackexchange.com/}{\TeX{} stackexchange} and \href{https://www.overleaf.com/learn}{overleaf}. If you can't find a symbol try \href{http://detexify.kirelabs.org/classify.html}{Detexify}. For a more complete overview of \LaTeX{} consider the freely accessible book \href{https://tobi.oetiker.ch/lshort/lshort.pdf}{The Not So Short Introduction to \LaTeXe}. For complete guides on specific packages, look at their documentation on \href{https://ctan.org/}{CTAN}.
\section*{Challenge problem}
It is the year of 1981 and Alexander Grothendieck is writing a thesis on algebraic geometry. However because he is really bad with computers (or, as he calls it, they are ``evil and not worth his time'') his thesis is now handwritten and illegible! Can you help him in his \href{https://en.wikipedia.org/wiki/Pursuing_Stacks}{pursuit of stacks}?

\subsection*{Macros}
Because some expressions occur very often, you would like to save some effort by creating macros. \LaTeX{} macros are ways to call a snippet of code and paste it in the text. They eliminate much of the repetition and reduce effort in fixing mistakes (if you fix something in the macro, it is also fixed wherever you use that macro).
\subsubsection*{Tasks}
Create the following commands.
\begin{itemize}
\item \verb|\hot| that displays \texttt{Hot} as a function (you may use \verb|\DeclareMathOperator|).
\item \verb|\rnaturals| that displays \(\{1,2,\dotsc,r\}\) when used. Be careful: make sure your ellipsis is done correctly \(\{1,2,\dotsc,r\}\neq\{1,2,...,r\}\).
\item \verb|\ProjK| that takes 1 argument. For example, when used with the argument W it results in \(\ProjK{W}\).
\item \verb|\funciso| that takes two parameters, one optional. When called with both parameters \verb|\funciso[g]{5}| it displays \(\funciso[g]{5}\). When the optional parameter is missing, it defaults to \(F\). For example \verb|\funciso{3}| displays \(\funciso{3}\)
\end{itemize}
Using the above macros, recreate the following snippet (you may use an environment such as \texttt{align*} or \texttt{gather*})
\begin{gather*}
  M\to(\hot)\\
  I \subset \rnaturals\\
  \Gamma = \ProjK{X}/(F_1,\dotsc,F_r)
\end{gather*}

\subsection*{Tables}
Help organize some theorems in a table to make them more readable! The suggested way to make a table in \LaTeX{} is using the booktabs library.
\subsubsection*{Tasks}
Recreate the following table.
\begin{table}[h]
    \begin{tabular}{c c c}\toprule
      Category type & Commutes & Size\\\midrule
      Weak test & Yes & 3\\
      Test & Yes & 2\\
      Strict test & No & 1\\\bottomrule
    \end{tabular}
  \centering
\end{table}

\subsection*{Graphs and diagrams}
Using the tikz-cd library (tikzcd environment) we can create commutative diagrams. In Gro\-then\-dieck's paper, there's a commutative diagram describing homotopy of certain functions. Additionally, numerous libraries allow us to plot functions easily. While not abstract enough for Alexander, we would like to see some plots of polynomials since that's what he's supposed to be writing about.
\subsubsection*{Tasks}
Create the following commutative diagram.
\begin{center}
\begin{tikzcd}
  F\times I \arrow[r, hookleftarrow] \arrow[d, hookleftarrow] \arrow[dr, dashrightarrow, "h"]
  & \funciso{1} \arrow[d, "f_1"]\\
  \funciso{0} \arrow[r, "f_0"]
  & G
\end{tikzcd}
\end{center}
Plot the function \(f(x) = 2x^3-3x-1\).
\begin{figure}[H]
  \centering
  \begin{tikzpicture}
    \begin{axis}[
      title=Example of a plot,
      xmin=-5,xmax=5,
      ymin=-5,ymax=5,
      axis x line=middle,
      axis y line=middle,
      axis line style=<->,
      xlabel={$x$},
      ylabel={$y$},
      ]
      \addplot[no marks] expression[domain=-5:5,samples=100]{2*x^3-3*x-1};
      \end{axis}
  \end{tikzpicture}
\end{figure}

\section*{Points to note}
Submit a compilable \LaTeXe{} code of the relevant parts of the text, each in the section their instructions come from (except for macros, which have to be written in the preamble (before the body of the text)). You can use any built-in library, but aren't allowed to bundle images. You can use any library, not only those mentioned, as long as the results are equivalent (or better).
\end{document}